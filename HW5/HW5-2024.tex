\documentclass[12pt, a4paper, oneside]{ctexart}
\usepackage{amsmath, amsthm, amssymb, bm, graphicx, hyperref, mathrsfs}

\title{\textbf{Homework}}
\author{PB22010344 黄境}
\date{\today}
\linespread{1.5}
\newcounter{exercisename}
\newenvironment{exercise}{\stepcounter{exercisename}\par\noindent\textsc{Exercise \arabic{exercisename}. }}{\\\par}
\newenvironment{solution}{\par\noindent\textsc{Solution. }}{\\\par}

\DeclareMathOperator*{\argmin}{\bf argmin\,}
\DeclareMathOperator*{\relint}{\bf relint\,}
\DeclareMathOperator*{\dom}{\bf dom\,}
\DeclareMathOperator*{\intp}{\bf int\,}
\DeclareMathOperator*{\prox}{\bf prox\,}

\begin{document}

\maketitle

\begin{exercise}
    \bf Proximal Operator
\end{exercise}

\begin{solution}
    1. We have
    \begin{align*}
	    p(\mathbf{x}_c) & = \argmin_\mathbf{x} \{g(\mathbf{x}) + \frac{L}{2} \|\mathbf{x} - (\mathbf{x}_c - \frac{1}{L} \nabla f(\mathbf{x}_c))\|^2\} \\
	    & = \argmin_\mathbf{x} \{\frac{g(\mathbf{x})}{L} + \frac{1}{2} \|\mathbf{x} - (\mathbf{x}_c - \frac{1}{L} \nabla f(\mathbf{x}_c))\|^2\} \\
	    & = \operatorname{prox}_{\frac{g}{L}} (\mathbf{x}_c - \frac{1}{L} \nabla f(\mathbf{x}_c))
    \end{align*}
    2.(a) $\forall\, \mathbf{x} \in \mathbb{R}^n$, $\operatorname{prox}_f(x)$ exists and is unique if and only if the optimization problem
    \[
    \min_{\mathbf{u} \in \dom f} f(\mathbf{u}) + \frac{1}{2} \|\mathbf{x} - \mathbf{u}\|^2
    \]
    has one unique solution. \newline
    Since $f$ is convex and close, $f(\mathbf{u}) + \frac{1}{2} \|\mathbf{x} - \mathbf{u}\|^2$ is strongly convex with parameter $1 > 0$. Therefore, from exercise 1.6 in HW4, the problom admits a unique solution, which implies that $\operatorname{prox}_f(x)$ exists and is unique, $\forall\, \mathbf{x} \in \mathbb{R}^n$.
    \newline\newline
    (b) Let $g(\mathbf{u}) = f(\mathbf{u}) + \frac{1}{2} \|\mathbf{x} - \mathbf{u}\|^2$, we have
    \[
    \partial g(\mathbf{u}) = \{\mathbf{g}: \mathbf{g} = \mathbf{g}_f + \mathbf{u} - \mathbf{x},\, \mathbf{g}_f \in \partial f(\mathbf{u})\}.
    \]
    ($\Rightarrow$) Since $\mathbf{u} = \argmin\limits_{\mathbf{u} \in \dom f} g(\mathbf{u})$, which implies that
    \[
    g(\mathbf{v}) \geq g(\mathbf{u}) = g(\mathbf{u}) + \langle\mathbf{0},\, \mathbf{u} - \mathbf{v}\rangle,\, \forall\, \mathbf{v} \in \dom f \quad \Rightarrow \quad \mathbf{0} \in \partial g(\mathbf{u})
    \]
    That is, $\mathbf{x} - \mathbf{u} \in \partial f(\mathbf{u})$ \newline
    ($\Leftarrow$) Similarly, since $\mathbf{x} - \mathbf{u} \in \partial f(\mathbf{u})$, we have $\mathbf{0} \in \partial g(\mathbf{u})$, which implies that $\mathbf{u} = \argmin\limits_{\mathbf{u} \in \dom f} g(\mathbf{u}) = \operatorname{prox}_f (\mathbf{x})$.
    \newline\newline
    3.(a) Let $\mathbf{v} = \lambda \mathbf{u} + \mathbf{a}$, we have
    \begin{align*}
    	\operatorname{prox}_h (\mathbf{x}) & = \argmin_{\mathbf{u} \in \dom h} \{h(\mathbf{u}) + \frac{1}{2} \|\mathbf{u} - \mathbf{x}\|^2\} \\
    	& = \argmin_{\lambda \mathbf{u} + \mathbf{a} \in \dom f} \{f(\lambda \mathbf{u} + \mathbf{a}) + \frac{1}{2} \| \mathbf{u} - \mathbf{x} \|^2\} \\
    	& = \frac{1}{\lambda} (\argmin_{\mathbf{v} \in \dom f} \{f(\mathbf{v}) + \frac{1}{2} \|\frac{\mathbf{v} - \mathbf{a}}{\lambda} - \mathbf{x}\|^2\} - \mathbf{a}) \\
    	& = \frac{1}{\lambda} (\argmin_{\mathbf{v} \in \dom f} \{\lambda^2 f(\mathbf{v}) + \frac{1}{2} \|\mathbf{v} - \mathbf{a} - \lambda \mathbf{x}\|^2\} - \mathbf{a}) \\
    	& = \frac{1}{\lambda} (\operatorname{prox}_{\lambda^2 f} (\lambda \mathbf{x} + \mathbf{a}) - \mathbf{a})
    \end{align*}
    \newline
    (b) Similarly,
    \begin{align*}
        \operatorname{prox}_h (\mathbf{x}) & = \argmin_{\mathbf{u} \in \dom h} \{h(\mathbf{u}) + \frac{1}{2} \|\mathbf{u} - \mathbf{x}\|^2\} \\
        & = \argmin_{\frac{\mathbf{u}}{\lambda} \in \dom f} \{\lambda f(\frac{\mathbf{u}}{\lambda}) + \frac{1}{2} \|\mathbf{u} - \mathbf{x}\|^2\} \\
        & = \frac{1}{\lambda} \argmin_{\mathbf{v} \in \dom f} \{\lambda f(\mathbf{v}) + \frac{1}{2} \|\lambda \mathbf{v} - \mathbf{x}\|^2\} \\
        & = \frac{1}{\lambda} \cdot \lambda^2 \argmin_{\mathbf{v} \in \dom f} \{\lambda^{-1} f(\mathbf{v}) + \frac{1}{2} \|\mathbf{v} - \frac{\mathbf{x}}{\lambda}\|^2\} \\
        & = \lambda \operatorname{prox}_{\lambda^{-1} f} (\frac{\mathbf{x}}{\lambda})
    \end{align*}
    \newline
    (c) 
    \begin{align*}
        \operatorname{prox}_h (\mathbf{x}) & = \argmin_{\mathbf{u} \in \dom f} \{ f(\mathbf{u}) + \mathbf{a}^T\mathbf{u} + \frac{1}{2} \|\mathbf{u} - \mathbf{x}\|^2\} \\
        & = \argmin_{\mathbf{u} \in \dom f} \{f(\mathbf{u}) + \frac{1}{2} \|\mathbf{u - x + a}\|^2 + \mathbf{a}^T\mathbf{x} - \frac{1}{2}\|\mathbf{a}\|^2\} \\
        & = \argmin_{\mathbf{u} \in \dom f} \{f(\mathbf{u}) + \frac{1}{2} \|\mathbf{u - x + a}\|^2\} \\
        & = \operatorname{prox}_f (\mathbf{x - a})
    \end{align*}
    4. (a) Consider two cases: \newline
    (i) $\mathbf{x} = \mathbf{0}$, and $\operatorname{prox}_f (\mathbf{x}) = \argmin\limits_{\mathbf{u} \in \dom f} \{ \|\mathbf{u}\| + \frac{1}{2} \|\mathbf{u}\|^2\} = \mathbf{0}$. \newline
    (ii) $\mathbf{x} \neq \mathbf{0}$. Fix $\|\mathbf{u}\| = t$, let
    \[
    g_t(\mathbf{x}) = \min_{\substack{\mathbf{u} \in \dom f \\ \|\mathbf{u}\| = t}} \{t + \frac{1}{2} \|\mathbf{u} - \mathbf{x}\|^2\},
    \]
    we have 
    \begin{align*}
	    g_t(\mathbf{x}) & = t + \frac{1}{2} \|\frac{t\mathbf{x}}{\|\mathbf{x}\|} - \mathbf{x}\|^2 \\
	    & = t + \frac{1}{2}(t - \|\mathbf{x}\|)^2 \\
	    & = \frac{1}{2}(t^2 + 2 t (1 - \|\mathbf{x}\|) + \|\mathbf{x}\|^2)
    \end{align*}
    Therefore
    \[
    \argmin\limits_{t} g_t(\mathbf{x}) = \|\mathbf{x}\| - 1 \quad \Rightarrow \quad \operatorname{prox}_f (\mathbf{x}) = \frac{\|\mathbf{x}\| - 1}{\|\mathbf{x}\|} \mathbf{x}.
    \]
    Since $t \geq 0$, we can conclude that
    \[
    \operatorname{prox}_f (\mathbf{x}) = 
    \left\lbrace
    \begin{aligned}
        & \frac{\|\mathbf{x}\| - 1}{\|\mathbf{x}\|} \mathbf{x},\, \mathbf{x} > 1 \\
        & \mathbf{0},\, \|\mathbf{x}\| \leq 1
    \end{aligned}
    \right.
    \]
    \newline
    (b) Let 
    \[
    g(\mathbf{u}) = I_C(\mathbf{u}) + \frac{1}{2} \|\mathbf{u} - \mathbf{x}\|^2, 
    \]
    we have
    \[
    g(\mathbf{u}) = 
    \left\lbrace
    \begin{aligned}
     	& 1 + \frac{1}{2} \|\mathbf{u} - \mathbf{x}\|^2,\, \mathbf{u} \in C \\
     	& +\infty,\, \mathbf{u} \notin C
    \end{aligned}
    \right.
    \]
    Therefore $\operatorname{prox}_f (\mathbf{x}) = \argmin\limits_{\mathbf{u} \in C} \|\mathbf{u} - \mathbf{x}\|^2 = \mathbf{P}_C(\mathbf{x})$ is the projection of $\mathbf{x}$ on C.
\end{solution}

\begin{exercise}
	\bf Proximal Gradient
\end{exercise}

\begin{solution}
	1. Since $\mathbf{x} \in \intp(\dom F)$, $\partial F(\mathbf{x})$ is nonempty. Therefore, we have
	\[
	F(\mathbf{y}) \geq F(\mathbf{x}) + \langle\mathbf{g}, \mathbf{y} - \mathbf{x}\rangle,\, \forall \mathbf{y} \in \dom F,
	\]
	which implies that
	\[
	F(\mathbf{y}) - F(\mathbf{x}) \geq \langle \mathbf{g}, \mathbf{y} - \mathbf{x} \rangle \geq 0 ,\, \forall \mathbf{y} \in \dom F,
	\]
	that is, $\mathbf{x}$ is optimal.
	\newline\newline
	3. Consider the situation that $\mathbf{x} \in \partial \dom F$. To simplify the question, let $F(x): \mathbb{R} \rightarrow \mathbb{R}$, we have
	\[
	\partial F(x) = \emptyset \quad \Leftrightarrow \quad \forall\, g \in \mathbb{R},\, \exists\, y \in \dom F,\, \text{ s.t. } F(y) < F(x) + g(y - x).
	\]
	Thus, we can set $F(y) = -e^y$, $\dom F = \mathbb{R}^+ \cup \{ 0 \}$, $x = 0$. \newline
	4. Since $f$ is twice continuously differentiable, by the Taylor's Theorem with Lagrange's form of remainder, we have
	\[
	f(\mathbf{y}) = f(\mathbf{x}) + \mathbf{J}f(\mathbf{x}) \cdot (\mathbf{y - x}) + \frac{1}{2} (\mathbf{y - x})^T \mathbf{H}f(\boldsymbol{\xi}) (\mathbf{y - x}),\, \boldsymbol{\xi} \in \overline{\mathbf{xy}}.
	\]
	Since $\mathbf{x}^T\mathbf{Hx} \leq \lambda_{max}\|\mathbf{x}\|^2$, where $\lambda_{max}$ is the largest eigenvalue of the symmetric matrix $\mathbf{H}$, we have
	\begin{align*}
		f(\mathbf{y}) & \leq f(\mathbf{x}) + \langle\nabla f,\, \mathbf{y - x} \rangle + \frac{1}{2} \lambda_{max} \|\mathbf{y - x}\|^2 \\
		& \leq f(\mathbf{x}) + \langle\nabla f,\, \mathbf{y - x} \rangle + \frac{L}{2} \|\mathbf{y - x}\|^2
	\end{align*}
	5. From exercise 1.6(d) in HW4, we only need to prove that $Q(\mathbf{x}; \mathbf{x}_c)$ is strongly convex with the parameter $L$, i.e. $g(\mathbf{x}) + \langle\nabla f(\mathbf{x}),\, \mathbf{x}\rangle$ is convex, which is obvious. \newline\newline
	6. $g(\mathbf{w}) = \lambda \|\mathbf{w}\|_1$, $f(\mathbf{w}) = \frac{1}{n}\|\mathbf{y} - X\mathbf{w}\|_2^2$, therefore, 
	\[
	\mathbf{w}^+ = p(\mathbf{w}_k) = \argmin_\mathbf{w} \{\lambda \|\mathbf{w}\|_1 + \frac{L}{2} \|\mathbf{w} - (\mathbf{w}_k - \frac{1}{L} \nabla f(\mathbf{w}_k))\|^2\}.
	\]
	Since $\mathbf{0} \in \partial p(\mathbf{w}^+)$, we have
	\[
	\mathbf{0} \in \partial \lambda\|\mathbf{w}^+\|_1 + \frac{L}{2} \nabla\|\mathbf{w^+ - z}\|^2 \quad \Rightarrow \quad \frac{L}{\lambda} (\mathbf{z - w^+}) \in \partial \lambda\|\mathbf{w}^+\|_1.
	\]
	Since
	\[
	\partial \lambda\|\mathbf{w}^+\|_1 = 
	\left\{ \mathbf{v} \in \mathbb{R}^n, v_i = 
	\left\lbrace
    \begin{aligned}
    	& 1,\, w_i > 0 \\
    	& [-1, 1],\, w_i = 0 \\
    	& -1,\, w_i < 0 \\
    \end{aligned}
    \right. 
    \right\},
	\]
	we have
	\[
	w_i^+ = 
	\left\lbrace
    \begin{aligned}
    	& z_i + \frac{\lambda}{L},\, z_i < -\frac{\lambda}{L} \\
    	& 0,\, |z_i| \leq \frac{\lambda}{L} \\
    	& z_i - \frac{\lambda}{L},\, z_i > \frac{\lambda}{L} \\
    \end{aligned}
    \right.
	\]
\end{solution}

\begin{exercise}
	\bf ISTA with Backtracking
\end{exercise}

\begin{solution}
	1. Let $\mathbf{x}_k = p_{L_k}(\mathbf{x}_{k - 1})$, we have
	\[
	\mathbf{0} \in \partial Q_{L_k}(\mathbf{x}_k;\, \mathbf{x}_{k - 1}) \Rightarrow -\nabla f(\mathbf{x}_k) - L_k (\mathbf{x}_k - \mathbf{x}_{k - 1}) \in \partial g(\mathbf{x}_k).
	\]
	Denote $-\nabla f(\mathbf{x}_k) - L_k (\mathbf{x}_k - \mathbf{x}_{k - 1})$ by $\mathbf{g}$. Therefore, 
	\[
	g(\mathbf{y}) - g(\mathbf{x}_k) \geq \langle\mathbf{g},\, \mathbf{y - x}_k\rangle,\, \forall\, \mathbf{y} \in \dom g.
	\]
	Since $f$ is convex, we have
	\[
	f(\mathbf{y}) \geq f(\mathbf{x}_{k - 1}) + \langle\nabla f(\mathbf{x}_{k - 1}),\, \mathbf{y - x}_{k - 1}\rangle,\, \forall\, \mathbf{y} \in \dom f.
	\]
	Set $\mathbf{y} = \mathbf{x}_k$, all together we obtain
	\begin{align*}
		F(\mathbf{x}_{k - 1}) - F(\mathbf{x}_k) & \geq  F(\mathbf{x}_{k - 1}) - Q_{L_k}(\mathbf{x}_k;\, \mathbf{x}_{k - 1}) \\
		& = f(\mathbf{x}_{k - 1}) - f(\mathbf{x}_{k - 1}) + g(\mathbf{x}_{k - 1}) - g(\mathbf{x}_k) \\
		& \quad + \langle\nabla f(\mathbf{x}_{k - 1}),\, \mathbf{x}_{k - 1} - \mathbf{x}_k\rangle - \frac{L_k}{2}\|\mathbf{x}_{k - 1} - \mathbf{x}_k\|^2 \\
		& \geq \langle\mathbf{g} + \nabla f(\mathbf{x}_{k - 1}),\, \mathbf{x}_{k - 1} - \mathbf{x}_k\rangle - \frac{L_k}{2}\|\mathbf{x}_{k - 1} - \mathbf{x}_k\|^2 \\
		& = L_k \langle\mathbf{x}_{k - 1} - \mathbf{x}_k,\, \mathbf{x}_{k - 1} - \mathbf{x}_k\rangle - \frac{L_k}{2}\|\mathbf{x}_{k - 1} - \mathbf{x}_k\|^2 \\
		& = \frac{L_k}{2}\|\mathbf{x}_{k - 1} - \mathbf{x}_k\|^2 \geq 0
	\end{align*}
	which implies that $F(\mathbf{x}_k)$ is non-increasing.
	\newline\newline
	2. Since
	\[
	f(\mathbf{x}_k) \leq f(\mathbf{x}_{k - 1}) + \langle\nabla f(\mathbf{x}_{k - 1}),\, \mathbf{x}_k - \mathbf{x}_{k - 1}\rangle + \frac{L}{2}\|\mathbf{x}_{k - 1} - \mathbf{x}_k\|^2,
	\]
	where $\|\mathbf{x}_{k - 1} - \mathbf{x}_k\|^2 \geq 0$, we have
	\begin{align*}
		F_{p_{\widetilde{L}}}(\mathbf{x}_{k - 1}) & = f(\mathbf{x}_k) + g(\mathbf{x}_k) \\
		& \leq f(\mathbf{x}_{k - 1}) + \langle\nabla f(\mathbf{x}_{k - 1}),\, \mathbf{x}_k - \mathbf{x}_{k - 1}\rangle + \frac{L}{2}\|\mathbf{x}_{k - 1} - \mathbf{x}_k\|^2 + g(\mathbf{x}_k) \\
		& \leq f(\mathbf{x}_{k - 1}) + \langle\nabla f(\mathbf{x}_{k - 1}),\, \mathbf{x}_k - \mathbf{x}_{k - 1}\rangle + \frac{\widetilde{L}}{2}\|\mathbf{x}_{k - 1} - \mathbf{x}_k\|^2 + g(\mathbf{x}_k) \\
		& = Q_{\widetilde{L}}(\mathbf{x}_k;\, \mathbf{x}_{k - 1}),\, \forall\, \widetilde{L} \geq L
	\end{align*}
	Therefore, inequality (3) is satisfied for any $\widetilde{L} \geq L$. \newline
	Moreover, $L_{k} = \eta^{\sum_{i = 1}^{k}i_k}L_0 = \eta^{j_k}L_0$, where $j_k = \sum_{i = 1}^{k}i_k$ is the smallest integer which satisfies inequality (3). Set $r_k = [\frac{\ln L - \ln L_0}{\ln \eta}] + 1 > 1$, we have 
	\[
	L \leq \eta^{r_k} L_0 \leq \eta L,
	\]
	which implies that $\eta^{r_k} L_0$ satisfies inequality (3). Since $j_k$ is the smallest, we have $j_k \leq r_k \Rightarrow L_{k} \leq \eta^{r_k} L_0 \leq \eta L$.
	\newline\newline
	3. From exercise 3.1, we have
	\[
	F(\mathbf{x}^*) - F(\mathbf{x}_k) \geq \frac{L_k}{2} (\|\mathbf{x}^* - \mathbf{x}_k\|^2 - \|\mathbf{x}^* - \mathbf{x}_{k - 1}\|^2).
	\]
	Since $F(\mathbf{x}_k)$ is non-increasing and $L_k \leq \eta L$,  summing up and we obtain
	\begin{align*}
		\frac{2k}{\eta L} (F(\mathbf{x}_k) - F(\mathbf{x}^*)) & \leq \sum_{i = 1}^{k} \frac{2}{\eta L} (F(\mathbf{x}_i) - F(\mathbf{x}^*)) \\
		& \leq \sum_{i = 1}^{k} \frac{2}{L_i} (F(\mathbf{x}_i) - F(\mathbf{x}^*)) \\
		& \leq \sum_{i = 1}^{k} \|\mathbf{x}^* - \mathbf{x}_{i - 1}\|^2 - \|\mathbf{x}^* - \mathbf{x}_i\|^2 \\
		& = \|\mathbf{x}^* - \mathbf{x}_0\|^2 - \|\mathbf{x}^* - \mathbf{x}_k\|^2 \\
		& \leq \|\mathbf{x}^* - \mathbf{x}_0\|^2
	\end{align*}
	Therefore, we have
	\[
	F(\mathbf{x}_k) - F(\mathbf{x}^*) \leq \frac{\eta L}{2k} \|\mathbf{x}^* - \mathbf{x}_0\|^2
	\]
\end{solution}

\begin{exercise}
	\bf Naive Bayes Classifier
\end{exercise}

\begin{solution}
	3. We can convert the product to a sum by calculating the logarithm to avoid data overflow.
	\newline\newline
	4. The result is shown below.
	\begin{figure}[htbp]
				\centering
			    \includegraphics[width=0.6\textwidth]{1.png}
	\end{figure}
	\newline\newline
	5. The result is shown below. We can see that Laplace smoothing technique is useful.\newline
	\begin{figure}[htbp]
				\centering
				\includegraphics[width=0.6\textwidth]{2.png}
	\end{figure}
\end{solution}

\begin{exercise}
	\bf Logistic Regression and Newton's Method
\end{exercise}

\begin{solution}
	1.(a) Since
	\[
	L(\mathbf{w}) = \frac{1}{n} (\sum_{i \in I^+} \ln(1 + e^{-\mathbf{w}^T\bar{\mathbf{x}}_i}) + \sum_{i \in I^-} \ln(1 + e^{\mathbf{w}^T\bar{\mathbf{x}}_i})),
	\]
	set $\mathbf{w}_n = n\hat{\mathbf{w}}$, $n \in \mathbb{N}^+$, we have
	\[
	\langle\mathbf{w}_n,\, \bar{\mathbf{x}}_i\rangle =  n\langle\hat{\mathbf{w}},\, \bar{\mathbf{x}}_i\rangle > 0,\, \forall\, i \in I^+, 
	\]
	\[
	\langle\mathbf{w}_n,\, \bar{\mathbf{x}}_i\rangle =  n\langle\hat{\mathbf{w}},\, \bar{\mathbf{x}}_i\rangle < 0,\, \forall\, i \in I^-, 
	\]
	therefore $L(\mathbf{w})$ is decreasing when $n \to \infty$, which implies that problem (4) has no solution on $\mathbb{R}^{d + 1}$
	\newline\newline
	(b) By the expression of $L(\mathbf{w})$ in exercise 4.1(a), we obtain that $L(\mathbf{w})$ is continuous on $\mathbb{R}^{d + 1}$. Therefore, if problem (4) has no solution, we must have 
	\[
	\lim\limits_{\|\mathbf{w}\| \to \infty} L(\mathbf{w}) = -\infty
	\]
	On the other hand, let $\mathbf{w}_0 \in \mathbb{R}^{d + 1}$, $\mathbf{w}_0 \neq \mathbf{0}$. WOLG, let $\langle\mathbf{w}_0,\, \bar{\mathbf{x}}_{i_0}\rangle < 0$, $i_0 \in I^+$. Therefore, we have
	\begin{align*}
		L(n\mathbf{w}_0) & = \frac{1}{n} (\sum_{i \in I^+} \ln(1 + e^{-n\mathbf{w}_0^T\bar{\mathbf{x}}_i}) + \sum_{i \in I^-} \ln(1 + e^{n\mathbf{w}_0^T\bar{\mathbf{x}}_i})) \\
		& \geq \ln(1 + e^{-n\mathbf{w}_0^T\bar{\mathbf{x}}_{i_0}}) \to \infty,\, n \to \infty,
	\end{align*}
	which leads to contradiction.
	\newline\newline
	2. Since $\nabla^2 L(\mathbf{w}) = \bar{\mathbf{X}}\mathbf{D}\bar{\mathbf{X}}^T$, where
	\[
	\mathbf{D} = diag(\frac{1 + e^{-\mathbf{w}^T\bar{\mathbf{x}}_1}}{(1 + e^{-\mathbf{w}^T\bar{\mathbf{x}}_1})^2},\, \frac{1 + e^{-\mathbf{w}^T\bar{\mathbf{x}}_2}}{(1 + e^{-\mathbf{w}^T\bar{\mathbf{x}}_2})^2},\, \dots,\, \frac{1 + e^{-\mathbf{w}^T\bar{\mathbf{x}}_n}}{(1 + e^{-\mathbf{w}^T\bar{\mathbf{x}}_n})^2},\,).
	\]
	Therefore, $\nabla^2 L(\mathbf{w})$ is positive definite, which implies that $L(\mathbf{w})$ is strictly convex.
\end{solution}

\begin{exercise}
	\bf Convergence of Stochastic Gradient Descent for Convex Function
\end{exercise}

\begin{solution}
	1. Since $F$ is strongly convex with parameter $\mu$, set 
	\[
	G(\mathbf{u},\, \mathbf{v}) = F(\mathbf{v}) + \langle\nabla F(\mathbf{v}),\, \mathbf{u - v}\rangle + \frac{\mu}{2}\|\mathbf{u - v}\|^2 \leq F(\mathbf{u}),
	\]
	we have 
	\[
	\nabla_\mathbf{u} G(\mathbf{u},\, \mathbf{v}) = \nabla F(\mathbf{v}) + \mu(\mathbf{u - v}),
	\]
	and
	\[
	\nabla^2_\mathbf{u} G(\mathbf{u}) = \mu I,
	\]
	which implies that $G(\mathbf{u},\, \mathbf{v})$ is convex, and
	\[
	\min\limits_{\mathbf{u}} G(\mathbf{u},\, \mathbf{v}) = 	G(\mathbf{v} - \frac{\nabla F(\mathbf{v})}{\mu},\, \mathbf{v}) = F(\mathbf{v}) - \frac{1}{2\mu} \|\nabla F(\mathbf{v})\|^2.
	\]
	Therefore, we have
	\[
	F^* = F(\mathbf{w}^*) \geq G(\mathbf{w}^*,\, \mathbf{w}) \geq F(\mathbf{w}) - \frac{1}{2\mu} \|\nabla F(\mathbf{w})\|^2,
	\]
	for all $\mathbf{w} \in \dom F$. Thus,
	\[
	F(\mathbf{w}) - F^* \leq \frac{1}{2\mu} \|\nabla F(\mathbf{w})\|^2.
	\]
	The strong convexity makes it easy to estimate the distance between $F(\mathbf{w}$ and $F^*$ with gradient.
	\newline\newline
	2. From Part 5.2 in Lecture 11, replace $\mathbf{g}(\xi_k) = f_{i_k}(\mathbf{w}_k)$ with $\mathbf{g}(\xi_k) = \frac{1}{n_m} \sum_{i \in \mathbf{S}_k}f_{i}(\mathbf{w}_k)$, we obtain
	\[
	\mathbb{E}_{\xi_k}[F(\mathbf{w}_{k + 1}) - F(\mathbf{w}_k)] \leq -\alpha(1 - \frac{L}{2}\alpha)\|\nabla F(\mathbf{w}_k)\|^2 + \frac{L}{2}\alpha^2 \mathbb{D}_{\xi_k}[\mathbf{g}(\xi_k)],
	\]
	since 
	\[
	\|\mathbb{E}[\mathbf{g}(\xi_k)]\|^2 = \|\frac{1}{n_m}\sum_{i \in \mathbf{S}_k}\mathbb{E}[f_i(\mathbf{w}_k)]\|^2 = \|\nabla F(\mathbf{w}_k)\|^2.
	\]
	With the Assumption 5, we have
	\begin{align*}
	\mathbb{D}_{\xi_k}[\mathbf{g}(\xi_k)] & = \mathbb{D}_{\xi_k}[\frac{1}{n_m} \sum_{i \in \mathbf{S}_k}f_{i}(\mathbf{w}_k)] \\
	& = \frac{1}{n_m^2} \sum_{i \in \mathbf{S}_k}\mathbb{D}_{\xi_k}[f_{i}(\mathbf{w}_k)] \\
	& = \frac{1}{n_m^2} \cdot n_m \mathbb{D}_{\xi_k}[f_{i}(\mathbf{w}_k)] \\
	& \leq \frac{1}{n_m}(M + M_V\|\nabla F(\mathbf{w})\|^2)
	\end{align*}
	Follow the same steps in Lemma 3, set $M_G = M_V + n_m$, we have 
	\[
	\mathbb{E}_{\xi_k}[F(\mathbf{w}_{k + 1}) - F(\mathbf{w}_k)] \leq -\alpha(1 - \frac{L}{2n_m}M_G\alpha)\|\nabla F(\mathbf{w}_k)\|^2 + \frac{L}{2n_m}M\alpha^2.
	\]
	Therefore, follow the same steps in Theorem 1, set $0 < \alpha < \frac{n_m}{LM_G}$, we have
	\[
	\mathbb{E}_{\xi_k}[F(\mathbf{w}_{k + 1}) - F^*] \leq F(\mathbf{w}_k) - F^* - \frac{\alpha}{2}\|\nabla F(\mathbf{w})\|^2 + \frac{L}{2n_m}M\alpha^2,
	\]
	combining with Lemma 4 leads to
	\[
	\mathbb{E}_{\xi_k}[F(\mathbf{w}_{k + 1}) - F^*] \leq (1 - \mu\alpha)(F(\mathbf{w}_k) - F^*) + \frac{L}{2n_m}M\alpha^2,
	\]
	that is,
	\[
	\mathbb{E}_{\xi_{k-1}}[F(\mathbf{w}_k) - F^* - \frac{LM}{2\mu n_m}\alpha] \leq (1 - \mu\alpha)(F(\mathbf{w}_{k-1}) - F^* - \frac{LM}{2\mu n_m}\alpha).
	\]
	Take the expectation with respect to $\xi_{k-1},\, \dots,\, \xi_0$, we obtain
	\[
	\mathbb{E}_{\xi_0:\xi_{k-1}}[F(\mathbf{w}_k) - F^* - \frac{LM}{2\mu n_m}\alpha] \leq (1 - \mu\alpha)^k(F(\mathbf{w}_0) - F^* - \frac{LM}{2\mu n_m}\alpha),
	\]
	which is the required inequality. \newline
	Since the speed of convergence in SGD is roughly $\frac{LM}{2}\alpha$, we can see that the number of step in SGD is roughly $n_m$ times than which mini-batch SGD needs.
\end{solution}

\end{document}