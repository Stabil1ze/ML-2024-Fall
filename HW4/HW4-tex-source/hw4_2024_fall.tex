%%%%%%%%%%%%%%%%%%%%%%%%%%%%%%%%%%%%%%%%%%%%%%%%%%%%%%%%%%%%%%%%
%
%  Template for homework of Introduction to Machine Learning.
%
%  Fill in your name, lecture number, lecture date and body
%  of homework as indicated below.
%
%%%%%%%%%%%%%%%%%%%%%%%%%%%%%%%%%%%%%%%%%%%%%%%%%%%%%%%%%%%%%%%%


\documentclass[11pt,letter,notitlepage]{article}

\usepackage[left=2cm, right=2cm, lines=45, top=0.8in, bottom=0.7in]{geometry}

\usepackage{fancyhdr}
\usepackage{fancybox}
\usepackage{graphicx}
\usepackage{pdfpages}
\usepackage{enumitem}
\usepackage{algorithm}
\usepackage{algorithmic}

\usepackage{epstopdf}
\usepackage{subfigure}
\usepackage{threeparttable}


\renewcommand{\headrulewidth}{1.5pt}
\renewcommand{\footrulewidth}{1.5pt}
\newcommand\Loadedframemethod{TikZ}
\usepackage[framemethod=\Loadedframemethod]{mdframed}

\usepackage{amssymb,amsmath}
\usepackage{amsthm}
\usepackage{thmtools}

\setlength{\topmargin}{0pt}
\setlength{\textheight}{9in}
\setlength{\headheight}{0pt}

\setlength{\oddsidemargin}{0.25in}
\setlength{\textwidth}{6in}

%%%%%%%%%%%%%%%%%%%%%%%%
%%%%%% Define math operator %%%%%
%%%%%%%%%%%%%%%%%%%%%%%%
\DeclareMathOperator*{\argmin}{\bf argmin}
\DeclareMathOperator*{\relint}{\bf relint\,}
\DeclareMathOperator*{\dom}{\bf dom\,}
\DeclareMathOperator*{\intp}{\bf int\,}
%%%%%%%%%%%%%%%%%%%%%%%


\setlength{\topmargin}{0pt}
\setlength{\textheight}{9in}
\setlength{\headheight}{0pt}

\setlength{\oddsidemargin}{0.25in}
\setlength{\textwidth}{6in}
\pagestyle{fancy}
%%%%%%%%%%%%%%%%%%%%%%%%
%% Define the Exercise environment %%
%%%%%%%%%%%%%%%%%%%%%%%%
\mdtheorem[
topline=false,
rightline=false,
leftline=false,
bottomline=false,
leftmargin=-10,
rightmargin=-10
]{exercise}{\textbf{Exercise}}
%%%%%%%%%%%%%%%%%%%%%%%
%% End of the Exercise environment %%
%%%%%%%%%%%%%%%%%%%%%%%


%%%%%%%%%%%%%%%%%%%%%%%%
%% Define the Problem environment %%
%%%%%%%%%%%%%%%%%%%%%%%%
\mdtheorem[
topline=false,
rightline=false,
leftline=false,
bottomline=false,
leftmargin=-10,
rightmargin=-10
]{problem}{\textbf{Problem}}
%%%%%%%%%%%%%%%%%%%%%%%
%% End of the Exercise environment %%
%%%%%%%%%%%%%%%%%%%%%%%

%%%%%%%%%%%%%%%%%%%%%%%
%% Define the Solution Environment %%
%%%%%%%%%%%%%%%%%%%%%%%
\declaretheoremstyle
[
spaceabove=0pt,
spacebelow=0pt,
headfont=\normalfont\bfseries,
notefont=\mdseries,
notebraces={(}{)},
headpunct={:\quad},
headindent={},
postheadspace={ },
postheadspace=4pt,
bodyfont=\normalfont,
qed=$\blacksquare$,
preheadhook={\begin{mdframed}[style=myframedstyle]\end{mdframed}},
	postfoothook={},
]{mystyle}

\declaretheorem[style=mystyle,title=Solution,numbered=no]{solution}
\mdfdefinestyle{myframedstyle}{%
	topline=false,
	rightline=false,
	leftline=false,
	bottomline=false,
	skipabove=-6ex,
	leftmargin=-10,
	rightmargin=-10}
%%%%%%%%%%%%%%%%%%%%%%%
%% End of the Solution environment %%
%%%%%%%%%%%%%%%%%%%%%%%

%% Homework info.
\newcommand{\posted}{\text{Nov. 6, 2024}}       			%%% FILL IN POST DATE HERE
\newcommand{\due}{\text{Nov. 14, 2024}} 			%%% FILL IN Due DATE HERE
\newcommand{\hwno}{\text{4}} 		           			%%% FILL IN LECTURE NUMBER HERE


%%%%%%%%%%%%%%%%%%%%
%% Put your information here %%
%%%%%%%%%%%%%%%%%%%
\newcommand{\name}{\text{San Zhang}}  	          			%%% FILL IN YOUR NAME HERE
\newcommand{\id}{\text{PBXXXXXXXX}}		       			%%% FILL IN YOUR ID HERE
%%%%%%%%%%%%%%%%%%%%
%% End of the student's info %%
%%%%%%%%%%%%%%%%%%%


\newcommand{\proj}[2]{\textbf{P}_{#2} (#1)}
\newcommand{\lspan}[1]{\textbf{span}  (#1)  }
\newcommand{\rank}[1]{ \textbf{rank}  (#1)  }
\newcommand{\RNum}[1]{\uppercase\expandafter{\romannumeral #1\relax}}
\DeclareMathOperator*{\cl}{\bf cl\,}
\DeclareMathOperator*{\bd}{\bf bd\,}
\DeclareMathOperator*{\conv}{\bf conv\,}
\DeclareMathOperator*{\epi}{\bf epi\,}
\DeclareMathOperator*{\argmax}{\bf argmax\,}

% \lhead{
% 	\textbf{\name}
% }
% \rhead{
% 	\textbf{\id}
% }
\chead{\textbf{
		Homework \hwno
}}




\begin{document}
\vspace*{-4\baselineskip}
\thispagestyle{empty}


\begin{center}
{\bf\large Introduction to Machine Learning}\\
{Fall 2024}\\
University of Science and Technology of China
\end{center}

\noindent
Lecturer: Jie Wang  			 %%% FILL IN LECTURER HERE
\hfill
Homework \hwno             			
\\
Posted: \posted
\hfill
Due: \due
% \\
% Name: \name             			
% \hfill
% ID: \id						
% \hfill

\noindent
\rule{\textwidth}{2pt}

\medskip





%%%%%%%%%%%%%%%%%%%%%%%%%%%%%%%%%%%%%%%%%%%%%%%%%%%%%%%%%%%%%%%%
%% BODY OF HOMEWORK GOES HERE
%%%%%%%%%%%%%%%%%%%%%%%%%%%%%%%%%%%%%%%%%%%%%%%%%%%%%%%%%%%%%%%%

\textbf{Notice, }to get the full credits, please present your solutions step by step.

\begin{exercise}[Convex Functions]
    \begin{enumerate}
        % \item (Optional)
        % For each of the following functions, determine whether it is convex. % and then show your conclusion.
        % % \item Judgement questions.
        % \begin{enumerate}
        %     \item
        %     $f(x)=x^2\log x$ on $\mathbb{R}_{++}$, where $\mathbb{R}_{++}=\{x\in\mathbb{R}:x>0\}$.
        %     \item
        %     $f(x_1,x_2)=x_1x_2$ on $\mathbb{R}^2$.
        %     \item
        %     $f(x_1,x_2)=\frac{x_1}{x_2}$ on $\mathbb{R}^2_{++}$, where $\mathbb{R}^2_{++}=\{ (x_1,x_2)\in\mathbb{R}^2:x_1,x_2>0\}$.
        %     \item
        %     $f(x_1,x_2)=\frac{x_1^2}{x_2}$ on $\mathbb{R}\times\mathbb{R}_{++}$.
        %     \item
        %     $f(x_1,x_2)=x_1^\alpha x_2^{1-\alpha}$ on $\mathbb{R}^2_{++}$, where $0\le\alpha\le 1$.
        % \end{enumerate}
        \item
        Please show that the following functions are convex.
        \begin{enumerate}
            % \item
            % $ f(\mathbf{x})=\log \sum_{i=1}^n e^{x_i}$
            % on $\dom f=\mathbb{R}^n$, where $x_i$ denotes the $i^{th}$ component of $\mathbf{x}$.

            \item
            $f(\mathbf{x})=\sum_{i=1}^k x_{[i]}$ on $\dom f=\mathbb{R}^n$, where $1\le k\le n$ and $x_{[i]}$ denotes the $i^{th}$ largest component of $\mathbf{x}$.

            % \item The extended-value extension of the indicator function of a convex set $C\subseteq\mathbb{R}^n$, i.e.,
            % \begin{align*}
            %     \Tilde{I}_c(\mathbf{x})=
            %     \begin{cases}
            %         0,\hspace{4mm}\mathbf{x}\in C,\\
            %         \infty,\hspace{3mm}\mathbf{x}\notin C.
            %     \end{cases}
            % \end{align*}

            \item
            The negative entropy, i.e.,
            $$
            f(\mathbf{p})=\sum_{i=1}^n p_i\log p_i
            $$
            on $\dom f=\{ \mathbf{p}\in\mathbb{R}^n:0<p_i\le 1, \sum_{i=1}^n p_i=1\}$, where $p_i$ denotes the $i^{\text{th}}$ component of $\mathbf{p}$.

            \item
            The $p$-norms, i.e., $ f(\mathbf{X})=\|\mathbf{X}\|_{p}$ on $\dom f=\mathbb{R}^{m\times n}$.

            % \item
            % $f(\mathbf{X})=\operatorname{tr} (\mathbf{X}^{-1})$ on $\dom f=\mathbb{S}_{++}^n$, where $\mathbb{S}_{++}^n$ is the space of all $n\times n$ real positive definite matrices.

        \end{enumerate}

    \item 
    % Please show the following two equalities:
    % % $$
    % % f(\mathbf{y})-f(\mathbf{x}) = \int_{0}^{1} \nabla f(\mathbf{x}+t(\mathbf{y}-\mathbf{x}))^{\top}(\mathbf{y}-\mathbf{x}) \mathrm{d} t \\
    % % $$
    % % $$
    % % \nabla f(\mathbf{y})-\nabla f(\mathbf{x}) = \int_{0}^{1} \nabla^{2} f(\mathbf{x}+t(\mathbf{y}-\mathbf{x}))(\mathbf{y}-\mathbf{x}) \mathrm{d} t
    % % $$
    % \begin{align}
    %     f(\mathbf{y})-f(\mathbf{x}) = \int_{0}^{1} \nabla f(\mathbf{x}+t(\mathbf{y}-\mathbf{x}))^{\top}(\mathbf{y}-\mathbf{x}) \mathrm{d} t \\
    %     \nabla f(\mathbf{y})-\nabla f(\mathbf{x}) = \int_{0}^{1} \nabla^{2} f(\mathbf{x}+t(\mathbf{y}-\mathbf{x}))(\mathbf{y}-\mathbf{x}) \mathrm{d} t
    % \end{align}
    % (\textbf{Hint:} you may consider the function $g(t)=f(\mathbf{x}+t(\mathbf{y}-\mathbf{x}))$ and apply the fundamental theorem of calculus.)
    Please show that a function $f$ is convex if and only if $\dom f$ is convex and its restriction to any line intersecting its domain is convex, i.e., for any $\mathbf{x}_{0} \in \dom f$ and $\mathbf{v} \in \mathbb{R}^{n}$, the function
    \begin{align*}
        g(t) = f(\mathbf{x}_{0} + t \mathbf{v})
    \end{align*}
    is convex over its domain $\dom g = \{t \in \mathbb{R}:\mathbf{x}_{0} + t \mathbf{v} \in \dom f \}$.
    
    (\textbf{Hint:} you may prove the sufficiency by contradiction.)

    \item (Optional)
    Please show that a continuously differentiable function $f$ is strongly convex with parameter $\mu>0$ if and only if
    \begin{align*}
        f(\mathbf{y})\ge f(\mathbf{x})+\langle\nabla f(\mathbf{x}),\mathbf{y}-\mathbf{x}\rangle+\frac{\mu}{2}\|\mathbf{y}-\mathbf{x}\|_2^2,\quad \forall\, \mathbf{x},\mathbf{y}\in\mathbb{R}^n.
    \end{align*}
    \item (Optional)
    Suppose that $f$ is twice continuously differentiable and strongly convex with parameter $\mu>0$. Please show that $\mu\leq \lambda_{\min}(\nabla^2 f(\mathbf{x}))$ for any $\mathbf{x}\in\mathbb{R}^n$, where $\lambda_{\min}(\nabla^2 f(\mathbf{x}))$ is the smallest eigenvalue of $\nabla^2 f(\mathbf{x})$.
    \item Suppose that $f:\mathbb{R}^n\rightarrow\mathbb{R}$ is twice continuously differentiable, and the gradient of $f$ is Lipschitz continuous, i.e.,
    \begin{align*}
        \|\nabla f(\mathbf{x})-\nabla f(\mathbf{y})\|_2\le L\|\mathbf{x}-\mathbf{y}\|_2,\quad \forall\,\mathbf{x},\mathbf{y}\in\mathbb{R}^n,
    \end{align*}
    where $L>0$ is the Lipschitz constant. Please show that $\lambda_{\max}(\nabla^2f(\mathbf{x}))\leq L$ for any $\mathbf{x}\in\mathbb{R}^n$, where $\lambda_{\max}(\nabla^2f(\mathbf{x}))$ is the largest eigenvalue of $\nabla^2 f(\mathbf{x})$.

    \item Consider the problem
    \begin{align}\label{prob:solution_of_conv}
        \min_{\textbf{x}\in\mathbb{R}^n}f(\mathbf{x}),
    \end{align}
    where $f:\mathbb{R}^n\rightarrow\mathbb{R}$ is  continuously differentiable and convex, and $\textbf{dom } f$ is closed.
    \begin{enumerate}
        \item
        Please show that the $\alpha$-sublevel set of $f$, i.e., $ C_\alpha=\{\mathbf{x}\in\textbf{dom }f:f(\mathbf{x})\leq \alpha\}
        $
        is closed.
        \item
        Please give an example to show that Problem (\ref{prob:solution_of_conv}) may be unsolvable even if $f$ is strictly convex.
        \item
        Suppose that $f$ can attain its minimum. Please show that the optimal set $\mathcal{C}=\{ \mathbf{y}:f(\mathbf{y})=\min_{\mathbf{x}} f(\mathbf{x})\}$ is closed and convex. Does this property still hold if $\textbf{dom }f$ is not closed?
        \item
        Suppose that $f$ is strongly convex with parameter $\mu>0$. Please show that Problem (\ref{prob:solution_of_conv}) admits a unique solution.
    \end{enumerate}
\end{enumerate}

\end{exercise}


\newpage
\begin{exercise}[ Operations that Preserve Convexity ]
    % \begin{enumerate}

    %     \item
        \begin{enumerate}
            \item Let $f:\mathbb{R}^m \rightarrow \left( -\infty,+\infty \right]$ be a given convex function, $\mathbf{A}\in \mathbb{R}^{m \times n}$ and $\mathbf{b} \in \mathbb{R}^m$. Please show that
        \begin{align*}
            F(\mathbf{x}) = f(\mathbf{Ax+b}),\quad\mathbf{x}\in\mathbb{R}^n.
        \end{align*}
        is convex.
        \item Let $f_i:\mathbb{R}^n \rightarrow \left(-\infty,+\infty \right],i=1,\dots,m$, be given convex functions. Please show that
        \begin{align*}
            F(\mathbf{x}) = \sum_{i=1}^m w_if_i(\mathbf{x})
        \end{align*}
        is convex, where $w_i \geq 0,\,i=1,\dots,m$.

        \item
            Let $f_i:\mathbb{R}^n \rightarrow \left(-\infty,+\infty \right]$ be given convex functions for $i \in I$, where $I$ is an arbitrary index set. Please show that the supremum
            \begin{align*}
                F(\mathbf{x}) = \sup_{i\in I}f_i(\mathbf{x})
            \end{align*}
            is convex.

        % \end{enumerate}


        % \item (Optional) Let A∈Rn×m,x0∈Rn\mathbf{A}\in \mathbb{R}^{n\times m},\mathbf{x}_0\in \mathbb{R}^n. The restriction of f:Rn→Rf:\mathbb{R}^n\rightarrow\mathbb{R} to the affine set {Az+x0|z∈Rm}\{\mathbf{Az}+\mathbf{x}_0|\mathbf{z}\in\mathbb{R}^m\} is defined as the function F:Rm→RF:\mathbb{R}^m\rightarrow\mathbb{R} with
        % \begin{align*}
        %     F(\mathbf{z})=f(\mathbf{Az}+\mathbf{x}_0)
        % \end{align*}
        % on dom F={z|Az+x0∈dom f}\textbf{dom }F=\{\mathbf{z}|\mathbf{Az}+\mathbf{x}_0\in\textbf{dom }f\}. Suppose ff is twice differentiable with a convex domain.
        % \begin{enumerate}
        %     \item Show that FF is convex if and only if for all z∈dom F\mathbf{z}\in\textbf{dom }F, we have
        %     \begin{align*}
        %         \mathbf{A}^\top\nabla^2f(\mathbf{Az}+\mathbf{x}_0)\mathbf{A}\succeq 0.
        %     \end{align*}
        %     \item Suppose B∈Rp×n\mathbf{B}\in\mathbb{R}^{p\times n} is a matrix whose nullspace is equal to the range of A\mathbf{A}, i.e., AB=0\mathbf{AB}=\mathbf{0} and rank(B)=n−rank(A)\operatorname{rank} (\mathbf{B})=n-\operatorname{rank}(\mathbf{A}). Show that FF is convex if for all z∈dom F\mathbf{z}\in\textbf{dom }F, there exists a λ∈R\lambda\in\mathbb{R} such that
        %     \begin{align*}
        %         \nabla^2f(\mathbf{Az}+\mathbf{x}_0)+\lambda\mathbf{B}^\top\mathbf{B}\succeq 0.
        %     \end{align*}

        %     (\textbf{Hint:} you can use the result as follows. If $\mathbf{C}\in\mathbb{S}^n$ and $\mathbf{D}\in\mathbb{R}^{p\times n}$, then $\mathbf{x}^\top\mathbf{C}\mathbf{x}\geq 0$ for all $\mathbf{x}\in\mathcal{N}(\mathbf{D})$ if there exists a $\lambda$ such that $\mathbf{C}+\lambda\mathbf{D}^\top\mathbf{D}\succeq 0.$)
        % \end{enumerate}


        % \item (Optional) \begin{enumerate}


        %     \item
        %     Consider the function $f(\mathbf{X})=\lambda_{\max}(\mathbf{X})$, with $\dom\, f =\mathbb{S}^n$, where $\lambda_{\max}(\mathbf{X})$ is the largest eigenvalue of $\mathbf{X}$ and $\mathbb{S}^n$ is the set of $n\times n$ real symmetric matrices. Show that $f$ is a convex function.
        %     \item
        %     Let $f:\mathbb{R}^n\to\mathbb{R}$ be a convex function, with $\textbf{dom }f=\mathbb{R}^n$. Show that it can be represented as the pointwise supremum of a family of affine functions, i.e.,
        %     $$
        %     f(\mathbf{x})=\sup\{ g(\mathbf{x}): g\text{ is affine},g(\mathbf{z})\le f(\mathbf{z}) \text{ for all }\mathbf{z}\in\mathbb{R}^n\}.
        %     $$
        % \end{enumerate}
        % \item
        % Suppose that the training set is {(xi,yi)}ni=1\{(\mathbf{x}_i,y_i)\}_{i=1}^n, where xi∈Rd\mathbf{x}_i\in\mathbb{R}^d is the ithi^{th} data instance and yi∈Ry_i\in\mathbb{R} is the corresponding label.
        % Recall that Lasso is the regression problem:
        % 	\begin{align*}
        %     	\min_{\mathbf{w}}\,\frac{1}{2n}\|\mathbf{X}\mathbf{w}-\mathbf{y}\|_2^2+\lambda\|\mathbf{w}\|_1,
        %     \end{align*}
        % 	where X∈Rn×d\mathbf{X}\in\mathbb{R}^{n\times d} with its ithi^{th} row being x⊤i\mathbf{x}_i^{\top}, w∈Rd\mathbf{w} \in \mathbb{R}^d, and λ>0\lambda>0 is the regularization parameter. Show that the objective function in the above problem is convex.
    	
    \end{enumerate}
\end{exercise}



\newpage
\begin{exercise}[Subdifferentials]
    Calculation of subdifferentials (you need to finish at least four of the problems).
        \begin{enumerate}
            \item Let $H\subset\mathbb{R}^n$ be a hyperplane. The extended-value extension of its indicator function $I_H$ is
            \begin{align*}
                \Tilde{I}_H(\textbf{x})=\begin{cases}
                    0,&\textbf{x}\in H,\\
                    \infty,&\textbf{x}\not\in H.
                \end{cases}
            \end{align*}
            Find $\partial \Tilde{I}_H(\textbf{x})$.

            \item Let $f(\textbf{x})=\exp{\|\textbf{x}\|_1},\, \textbf{x}\in\mathbb{R}^n$. Find $\partial f(\textbf{x})$.


            \item For $\textbf{x}\in\mathbb{R}^n$, let $x_{[i]}$ be the $i^{th}$ largest component of $\textbf{x}$. Find the subdifferentials of
            \begin{align*}
                f(\textbf{x})=\sum_{i=1}^k x_{[i]}.
            \end{align*}
            
            \item Let $f(\textbf{x})=\|\mathbf{x}\|_\infty,\, \textbf{x}\in\mathbb{R}^n$. Find $\partial f(\textbf{x})$.
            % TODO 改成 Ax + b 之类的函数,可以考虑换成其他常见的函数

            \item Let $f(X)=\max\limits_{1\le i\le n}|\lambda_i|$, where $X\in \mathbb{S}^n$ and $\lambda_1,\dots,\lambda_n$ are the eigenvalues of $X$. Find $\partial f(X)$.
            
            (\textbf{Hint}: you can refer to Example 7 in Lec06.)
            % TODO 课件有修改
     
    \end{enumerate}
\end{exercise}

\newpage
\bibliography{refs}
\bibliographystyle{abbrv}

%%%%%%%%%%%%%%%%%%%%%%%%%%%%%%%%%%%%%%%%%%%%%%%%%%%%%%%%%%%%%%%%

\end{document}
